\maketitle

\tableofcontents
\newpage

\section{Revision del paper}
% TODO: Poner tema, autor, intitucion en la que se realizo y año de publicacion
\subsection{Tema}
Optimización basada en el comportamiento inteligente de las plantas con sus
aplicaciones ingenieriles: Algoritmo Ivy

\subsection{Autores}
Mojtaba Ghasemi \textsuperscript{a}, Mohsen Zare \textsuperscript{b}, Pavel
Trojovský \textsuperscript{c}, Ravipudi Venkata Rao \textsuperscript{d},
Eva Trojovská \textsuperscript{c}, Venkatachalam Kandasamy \textsuperscript{c}

\subsection{Instituciones en la que se realizo}
\noindent\textsuperscript{a} Departamento de Electronica y Ingenieria Electrica, Universidad de Tecnologia de Shiraz, Shiraz, Iran \par
\noindent\textsuperscript{b} Departamento de Ingenieria Electrica, Facultad de Ingenierica, Jahrom University, Jahrom, Iran \par
\noindent\textsuperscript{c} Departamento de Mathematicas, Facultad de Ciencia, Universidad de kradec Králové, Rokitanského 62, Hradec Králové, Czech Republic \par
\noindent\textsuperscript{d} Departamento of Ingenieria Mecanica, Instituto Nacional de Technologia Sardar Vallabhbhai, Ichchanath, Gujarat, Surat, India \par

\subsection{Año de publicacion}
Julio 8 del 2024

\subsection{Identificacion del problema}
Los algoritmos bioinspirados son en muchas ocaciones ineficientes en cuestion
de rendimiento. Tomando mucho tiempo en obtener resultados esperados. Este
estudio presenta una variante poderosa y novedosa de modelado de algoritmos
bioinspirados, denominada algoritmo Ivy (IVYA), basada en los patrones de
crecimiento de las plantas de hiedra. El algoritmo utiliza el conocimiento de
plantas de hiedra cercanas para determinar la dirección de crecimiento. Las
características únicas del IVYA, como la preservación de la diversidad
poblacional, su simplicidad y flexibilidad, permiten una fácil modificación y
extensión, lo que habilita a investigadores y profesionales a explorar diversas
modificaciones y técnicas para mejorar su rendimiento y capacidades. Estos son
aspectos fundamentales en la optimización de problemas de ingeniería. El IVYA
se compara con diversos otros algoritmos, poniendo en prueba su rendimiento.

\subsection{Objetivos}
\begin{itemize}
    \item Comparar los resultados del optimizador Ivy con diez optimizadores
        metaheurísticos reconocidos, establecidos y novedosos, para ilustrar su
        rendimiento y demostrar las ventajas de desarrollar este algoritmo.\par
    \item Mostrar que, en contraste con los algoritmos comparativos, el enfoque
        propuesto exhibe una mayor velocidad de optimización y una complejidad
        computacional adecuada.\par
    \item Demostrar que el nuevo enfoque supera a muchos algoritmos anteriores
        mejorados en los resultados de optimización para funciones de prueba
        estándar.\par
    \item Evidenciar que el método propuesto posee mayor capacidad de
        optimización y se adapta mejor a diversas tareas de optimización en
        ingeniería, en comparación con otros algoritmos propuestos.\par
\end{itemize}

\subsection{Marco Teorico} % especifico del algoritmo bioinspirado
Este trabajo se basa en dos conceptos importantes: los algoritmos de
optimización basados en inteligencia artificial (IA) y la idea de que las
plantas también pueden mostrar comportamientos inteligentes. Ambos temas son
fundamentales para entender cómo se desarrolló el Ivy Algorithm (IVYA).

\subsubsection*{Algoritmos de Optimización Bioinspirados e Inteligencia Artificial}
Los algoritmos de optimización inspirados en IA buscan resolver problemas
complejos encontrando la mejor solución posible, de forma similar a como los
humanos o los sistemas naturales resuelven sus propios desafíos. Estos
algoritmos exploran el espacio de búsqueda con métodos como el aprendizaje
automático y la computación evolutiva. La gracia de estos enfoques es que se
pueden adaptar con el tiempo a diferentes condiciones y restricciones, lo que
los convierte en herramientas muy útiles en áreas como la ingeniería, las
finanzas y la medicina. En estos campos, la automatización de decisiones y la
optimización son fundamentales, y los algoritmos bioinspirados son una forma
eficaz de lograrlo.

\subsubsection{Inteligencia en Plantas como Inspiración para Algoritmos}
Últimamente, se ha estudiado mucho cómo las plantas "piensan" o reaccionan a su
entorno, y esto ha cambiado la percepción de que las plantas son pasivas. Las
plantas pueden responder a cambios en el ambiente, adaptarse, y hasta aprender
de ciertas experiencias, lo que algunos investigadores llaman "inteligencia en
plantas". Este concepto ha inspirado el diseño de varios algoritmos, incluyendo
el IVYA, que imita el comportamiento de las plantas trepadoras, como la hiedra.
Así como la hiedra se expande y se adapta a diferentes entornos, el algoritmo
simula estos mismos principios para mejorar la búsqueda de soluciones óptimas
en problemas complejos.

\subsubsection{Desafíos en Algoritmos Evolutivos}
En los algoritmos evolutivos tradicionales, uno de los grandes problemas es la
"convergencia prematura": cuando el algoritmo se "atasca" en una solución
subóptima sin seguir explorando. Para evitar esto, se suelen utilizar ciertos
términos que permiten mantener la diversidad en el conjunto de soluciones, lo
cual es fundamental para no caer en óptimos locales. En el IVYA, esto se aborda
con una ecuación de crecimiento que mantiene la diversidad de la "población"
durante la optimización. Es decir, el algoritmo evita quedarse "encerrado" en
una sola solución y sigue buscando otras posibilidades, lo que es especialmente
útil en problemas de ingeniería donde las restricciones suelen ser complejas.

\subsubsection{Aplicación del IVYA en Problemas de Ingeniería}
El IVYA fue diseñado específicamente para problemas de optimización en
ingeniería. Para manejar las restricciones de estos problemas, usa un enfoque
de penalización estática, como se muestra en la ecuación incluida en el
artículo. Básicamente, esta técnica permite que el algoritmo se adapte mejor a
las limitaciones impuestas por el problema, haciendo que el IVYA sea más
eficiente en su búsqueda de soluciones viables en aplicaciones reales.

\subsection{Resultados} % mas relevantes
l IVYA fue probado y comparado con diez algoritmos ampliamente conocidos en una
variedad de funciones de prueba, incluyendo funciones unimodales, multimodales,
variaciones clásicas y benchmarks con parámetros desplazados y rotados. En
estas pruebas, el IVYA mostró un rendimiento competitivo, superando a otros
algoritmos en varias de ellas. Esto confirma que el IVYA no solo alcanza una
alta velocidad de convergencia, sino que también maneja eficazmente la
diversidad poblacional, evitando así la convergencia prematura en óptimos
locales, una ventaja importante en problemas de optimización complejos.

Además, el IVYA demostró su capacidad para abordar problemas de optimización en
ingeniería, gestionando las restricciones típicas de estos problemas mediante
un enfoque de penalización estática. Esto sugiere un buen potencial de
aplicación en entornos reales, donde se requieren algoritmos que sean tanto
robustos como adaptables.

\subsection{Critica del paper} % acerca de sus conclusiones
El paper no trata de poner a prueba la eficacia de los resultados obtenidos por
el algoritmo, solo se enfoca en el rendimiento de esta.

\section{Informacion complementaria} % sobre el algoritmo bioinspirado

\subsection{Conceptos adicionales} % no considerados en el paper
% TODO: Poner, por ejemplo: Caracteristicas mas relevantes, ventajas y desventajas, etc
Un aspecto que no se recalco en el texto fue la adaptabilidad de los algoritmos
bioinspirados. Estos algoritmos tienen la capacidad de ajustarse a
cambios en el entorno y a nuevas condiciones, lo que los hace especialmente
útiles para resolver problemas dinámicos. Por ejemplo, los Algoritmos Genéticos
aplican principios de selección natural para evolucionar soluciones a lo largo
de múltiples generaciones, lo que permite una búsqueda más eficiente en
espacios de soluciones complejos. Además, muchos de estos algoritmos operan
dentro de un marco de procesamiento paralelo, lo que significa que pueden
evaluar múltiples soluciones simultáneamente. Esta capacidad no solo mejora la
eficiencia de la búsqueda, sino que también permite explorar un espacio de
soluciones más amplio y diverso.

En relación con la diversidad, los algoritmos bioinspirados suelen mantener un
equilibrio entre la exploración de nuevas soluciones y la explotación de
soluciones previamente identificadas. Este equilibrio es crucial para evitar
caer en óptimos locales, un problema común en algoritmos de optimización más
convencionales. Gracias a esta característica, los algoritmos bioinspirados son
particularmente eficaces en problemas donde las técnicas tradicionales podrían
ser inadecuadas, ofreciendo soluciones óptimas o casi óptimas en un tiempo
razonable. También se destaca su escalabilidad, lo que permite su aplicación en
una variedad de problemas, desde aquellos con pocos parámetros hasta los que
requieren el manejo de grandes volúmenes de datos.

No obstante, los algoritmos bioinspirados presentan ciertas desventajas. Uno de
los principales inconvenientes es su convergencia lenta, especialmente en
problemas complejos. Esto puede resultar en tiempos de ejecución prolongados,
lo que no siempre es deseable en aplicaciones que requieren respuestas rápidas.
Además, la eficacia de estos algoritmos a menudo depende de la adecuada
selección de parámetros. Por ejemplo, en el caso de los algoritmos genéticos,
la tasa de mutación es un factor crítico, y su calibración adecuada puede
resultar un desafío que exige pruebas exhaustivas.

Otra desventaja significativa es la incertidumbre en los resultados. Dado que
estos algoritmos incorporan elementos aleatorios en su proceso de búsqueda, los
resultados pueden variar entre diferentes ejecuciones. Esta variabilidad puede
ser problemática en aplicaciones críticas donde la consistencia y la
previsibilidad son esenciales.

\subsection{Otras aplicaciones} % en la que se puede utilizar el algoritmo bioinspirado
Debido a que el trabajo no se especificaba en una aplicacion especifica y es
mas generio. Este podria tener aplicaciones en casi todas las ramas de la
ciencia.

El algoritmo IVYA se desarrolla en un
contexto donde los algoritmos bioinspirados han demostrado ser útiles en una
variedad de aplicaciones de optimización. La literatura existente, como se
observa en estudios recientes que abordan la optimización mediante algoritmos
híbridos, sugiere que IVYA podría ser adaptado para resolver problemas en
múltiples dominios de optimizacion. Y varios trabajos ya citaron este paper.

Por ejemplo, el trabajo titulado "A hybrid butterfly and Newton–Raphson swarm
intelligence algorithm based on opposition-based learning" resalta la
efectividad de los enfoques híbridos en problemas complejos de optimización, lo
que indica que IVYA podría beneficiarse de estrategias similares. Esta
característica híbrida podría facilitar su aplicación en áreas como la
ingeniería mecánica y civil, donde se requiere la optimización de diseños
complejos, como estructuras y mecanismos, garantizando no solo eficiencia sino
también seguridad y funcionalidad.

Además, el DHRDE (Dual-population hybrid update and RPR mechanism based
differential evolutionary algorithm for engineering applications) demuestra la
eficacia de utilizar poblaciones duales para mejorar la convergencia en
problemas de optimización. Siguiendo esta línea, IVYA podría implementar
estrategias de actualización dual, permitiendo que el algoritmo explore y
explote soluciones de manera más efectiva. Esto podría ser particularmente útil
en la optimización de sistemas dinámicos donde las condiciones cambian con
frecuencia, como en la programación de sistemas de energía renovable, donde las
variaciones en la producción de energía requieren ajustes constantes en las
configuraciones.

El estudio "A novel balanced teaching-learning-based optimization algorithm for
optimal design of high efficiency plate-fin heat exchanger" también proporciona
una referencia interesante. La optimización de intercambiadores de calor es un
problema bien establecido en la ingeniería térmica, y IVYA podría aplicarse
aquí, aprovechando su capacidad para realizar búsquedas intensivas en espacios
de solución complejos. La combinación de técnicas de optimización puede
permitir una mejora en la eficiencia térmica y la reducción de costos
operativos.

Por último, la investigación sobre el SDO (Sled Dog Optimizer) sugiere que los
algoritmos inspirados en la naturaleza pueden ser altamente eficaces en el
tratamiento de problemas complejos en diversas áreas. Al igual que el SDO, IVYA
podría ser aplicado en la gestión de recursos naturales, como la planificación
de rutas en logística y distribución, donde se requiere la minimización de
costos y tiempos. Además, IVYA podría contribuir a la optimización en la
producción agrícola, mejorando la asignación de recursos y aumentando los
rendimientos.

En conclusión, el algoritmo IVYA no solo se fundamenta en las bases de los
algoritmos bioinspirados, sino que, al igual que los trabajos citados, ofrece
un marco prometedor para ser adaptado y aplicado en diversas áreas, incluyendo
la ingeniería, la gestión de recursos y la optimización de sistemas dinámicos.
Su flexibilidad y capacidad de adaptación lo convierten en una herramienta
valiosa para abordar los desafíos actuales en múltiples disciplinas.

\section*{Fuentes}
\label{sec:fuentes}
\addcontentsline{toc}{section}{\nameref{sec:fuentes}}

\begin{itemize}
    \item \url{https://www.scopus.com/record/display.uri?eid=2-s2.0-85191987789&origin=resultslist&sort=plf-f&src=s&sid=a5db478061e4769d651f5acd42d1e033&sot=b&sdt=cl&cluster=scoexactkeywords%2C%22Bio-inspired+Algorithms%22%2Ct&s=TITLE-ABS-KEY%28bio-inspired+algorithm%29&sl=37&sessionSearchId=a5db478061e4769d651f5acd42d1e033&relpos=12}
    \item \url{https://www.sciencedirect.com/science/article/abs/pii/S0950705124004842?via%3Dihub}
\end{itemize}

